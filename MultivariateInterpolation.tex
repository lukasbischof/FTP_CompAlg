\section{Multivariate Polynomial Interpolation}

\subsection{Bilinear Interpolation}

E.g.\ we want to interpolate color between four pixels.
We get the four basis functions $\{\pi_0(x), \pi_1(x)\} \times \{\pi_0(y), \pi_1(y)\}$
represented by a 2-fold tensor product:
\begin{align*}
    \overbrace{
        \left.
        \begin{bmatrix}
            1\cdot 1       & (x-x_0)\cdot 1 \\
            1\cdot (y-y_0) & (x-x_0)(y-y_0)
        \end{bmatrix}
        \right\}
    }^{\begin{matrix}
           \pi_0(x) & & & & \pi_1(x)
    \end{matrix}}
    \begin{matrix}
        \pi_0(y) \\
        \pi_1(y)
    \end{matrix}
\end{align*}

and the polynomial
\begin{align*}
    p(x,y) = &\ a_{0,0}\pi_0(x)\pi_0(y)
    + a_{1,0}\pi_1(x)\pi_0(y) \\
    \ & + a_{0,1}\pi_0(x)\pi_1(y)
    + a_{1,1}\pi_1(x)\pi_1(y)
\end{align*}

Then, the computation proceeds using divided differences in all steps:
\begin{enumerate}
    \item{
        Univariate interpolation for $x$ when holding $y=y_0$,
        ignoring a y parameter
        \begin{align*}
            p(x,y_0) = p(x_0,y_0)\pi_0(x) + \frac{p(x_n,y_0) - p(x_0, y_0)}{x_n - x_0}\pi_1(x)
        \end{align*}
        using the measurements at $x_0$ and $x_n$.
        We therefore receive a two-dimensional, univariate polynom
    }
    \item{
        Univariate interpolation for $x$ when holding $y=y_n$:
        \begin{align*}
            p(x,x_n) = p(x_0,y_n)\pi_0(x) + \frac{p(x_n,y_n) - p(x_0, y_n)}{x_n - x_0}\pi_1(x)
        \end{align*}
        giving a second polynomial ``at the other edge''
    }
    \item{
        Univariate interpolation for $y$:
        \begin{align*}
            p(x,y)=p(x,y_0)\pi_0(y)+\frac{p(x,y_1) - p(x,y_0)}{y_1-y_0}\pi_1(y)
        \end{align*}
        where we can substitute $p(x,y_1)$ and $p(x,y_0)$ with the above calculations
    }
\end{enumerate}
