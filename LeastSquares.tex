\section{Least-Squares Approximation}

\begin{itemize}
    \item Error = Residuum, errors = residuals. $e_i=r_i=y_i - \sum_{j=0}^ma_jg_j(x_i)$
    \item $\min_i\sum_i|e_i|$: $L^1$ norm
    \item $\min_i\sum_ie_i^2$: $L^2$ norm
    \item Basis functions $g_0,g_1,\ldots,g_m$ (m=degree, (polynom $x^m$))
    \item normalisation of data: $\frac{x-\mu}{\sigma}$ ($\mu$=mean, $\sigma$=deviation)
\end{itemize}

Some basis functions are:
\begin{itemize}
    \item{
        Trigonometric:
        \begin{align*}
            \left\{e^{i k x}\ |\ k\in Z\right\}{\text{or }}\left\{{\cos(k x){\mathrm{,}}\sin(k x)\ |\ k\in Z}\right\}\quad[x\in(0,2\pi)]
        \end{align*}
    }
    \item{
        Polynomials:
        \begin{align*}
            & \left\{x^{j}\ |\ j\in \mathbb{N}_{0}\right\} & \text{(standard monomials)} \\
            & \left\{\left({\frac{x-\mu}{\sigma}}\right)^{j}|\ j\in \mathbb{N}_{0}\right\} & \text{(normalised standard monomials)} \\
            & \left\{ \pi_{j}(x)\ |\ j\in \mathbb{N}_{0} \right\} & \text{(Newton polynomials)} \\
            & \left\{T_{j}(x)\ |\ j\in \mathbb{N}_{0}\right\} & \text{(Chebyshev, }x\in(-1,1))
        \end{align*}
        \begin{align*}
            & \left\{L_{j}(x)=\frac{e^{x}}{j!}\frac{d^{j}}{d x^{j}}\Big(e^{-x}x^{j}\Big)\ |\ j\in \mathbb{N}_{0}\right\}
            \text{(Laguerre, }x\in(0,\infty)) \\
            & \left\{H_{j}(x)=(-1)^{j}e^{\frac{x^{2}}{2}}\,{\frac{d^{j}}{d x^{j}}}\left(e^{-\frac{x^{2}}{2}}\right)\ |\ j\in \mathbb{N}_{0}\right\} \\
            &\quad\text{(Hermite, }x\in(0,\infty))
        \end{align*}
    }
    \item{
        Exponentials
        \begin{align*}
            \left\{ e^{\alpha kx}\ |\ k\in\mathbb{Z} \right\}
        \end{align*}
    }
\end{itemize}

\subsection{Design Matrix}
Given a \emph{measurement} with $N-1$ points $\{(x_i,y_i)\}_{i=0,\ldots,N}$ (with $x$ a vector in the multivariate case)
and a set of \emph{basis functions} $\{g_j\}_{j=0,\ldots,m}$,
the design matrix $G$ results from the linear system
$y_i=\sum_{j=0}^ma_jg_j(x_i)\quad(i=0,\ldots,N)$ with $N+1$ equations and $m+1$ unknowns:

\begin{snugshade*}
    \begin{align*}
        \underbrace{
            \begin{bmatrix}
                g_0(x_0) & \hdots & g_m(x_0) \\
                \vdots   & \ddots & \vdots   \\
                g_0(x_N) & \hdots & g_m(x_N)
            \end{bmatrix}
        }_{\text{Design matrix }G}
        \cdot
        \begin{bmatrix}
            a_0    \\
            \vdots \\
            a_m
        \end{bmatrix}
        =
        \begin{bmatrix}
            y_0    \\
            \vdots \\
            y_N
        \end{bmatrix}
    \end{align*}
\end{snugshade*}
that in general is overdetermined ($m<N$) thus the squared sum of residuals
$S=\sum_{i=0}^{N}\left(y_{i}-\sum_{j=0}^{m}a_{j}g_{j}(x_{i})\right)^{2}$
has to be minimised [$\downarrow$].

\paragraph{Normal Equation} Theorem: The squared sum of residuals $S$ is minimal iff:
\colorbox{shadecolor}{$
G^T\cdot G\cdot a| = G^T\cdot y|\quad (a|\text{ and }y|\text{ are the column vectors})
$}

with $G^T\cdot G$ being a normal,
symmetric and relatively small $(m+1)\times(m+1)$ matrix resulting in $f(x)=\sum a_jg_j(x)$ as best $L^2$ fit for data.

\subsection{Singular Value Decomposition}
Any real matrix $G$ of dimensions $((N+1)\times (m+1))$ can be decomposed as the product $G=UDV^T$
where $U$ is an orthogonal (meaning $UU^t=U^tU=I\Rightarrow U^{-1}=U^T$) (large) matrix with $\dim U = (N+1)\times (N+1)$,
D is a diagonal matrix with $\dim D = (N+1)\times (m+1)$ and
V is orthogonal (small) with $\dim V = (m+1)\times (m+1)$.

Using this decomposition, the minimisation of the square sum of residuals can be found in a numerically stable way:
With $G$ decomposed to $G=UDV^T$, the system translates to
\colorbox{shadecolor}{
    $\hat{r}| = \hat{y}| - D\hat{a}| = \vec{0}\quad$
        {\color{darkgray} $\left[\hat{r} = U^Tr|,\ \hat{y}| = U^Ty|\text{ and }\hat{a} = V^Ta|\right]$}
}

and is solved for zero since $\hat{S}$ is minimal iff $\hat{r}_i=0\quad(i=0,\ldots,m)$.
Due to the orthogonality of $U$, the square sum $\hat{S}$ of residuals $\hat{r}$ is equal to the square sum $S$ of the original
residuals $r|$.

In the final step, we receive $a| = V\cdot\hat{a}|$.

\subsection{Uniform Arguments and Orth. Polynomials}

If the set of basis functions $\{g_j\}_{j=0,\ldots,m}$ is \emph{orthogonal} with respect to the inner product
$\langle g_j,g_k\rangle := \sum_{i=0}^N g_j(x_i)g_k(x_i)$, $G^TG$ is diagonal and the solution is trivial:
\begin{align*}
    a_j=\frac{\langle y|, g_j|\rangle}{\langle g_j|, g_j|\rangle}\quad(j=0,1,\ldots,m)
\end{align*}

\paragraph{Uniform Arguments}
If we have a set of uniformly distributed arguments of a measurement $\{x_0 + t\cdot h\}_{t=0,\ldots,N}$,
then is the set of polynomials
\begin{align*}
    p_{k,N}(t)
    = \sum_{i=0}^{k}(-1)^{i}{\binom{k}{i}}{\binom{k+i}{i}}{\frac{t^{(i)}}{N^{(i)}}}\quad(k=0,1,...,N)
\end{align*}
with integer normalisation $t = (x-x_0)\div h$ orthogonal with respect to the inner product $\langle g,\tilde{g}\rangle$.
Here, $N^{(i)} = N(N-1)(N-2)\cdots(N-i+1)$ and $t^{(i)} = t(t-1)(t-2)\cdots(t-i+1)$ are defined as falling factorials.

$\Rightarrow$ We therefore get a design matrix with ``x''-values $p_{0,N}(t), p_{1,N}(t),\ldots$
(e.g. $1$; $1-\frac{t}{2}$; $1-\frac{3t}{2}+\frac{1}{2}(-1+t)t$, ...) and ``y''-values being the $t=0,\ldots,N$ integer values.

\subsection{Chebyshev Orthogonal Polynomials}
Analogously as in polynomial collocation, these polynomials help avoid Runge-phenomenon.

The T-polynomials are defined on the interval $[-1,1]$:
\begin{align*}
    T_n(x)=\cos(n\cdot \arccos(x))\quad (n=0,1,\ldots)
\end{align*}

The degree $n$ is the amount of divisions (intersections at $y=0$).
The first few Chebyshev polynomials are:
\begin{align*}
    & \overbrace{1}^{0};\overbrace{x}^{1};\overbrace{2x^2-1}^{2};\overbrace{4x^3-3x}^{3};\overbrace{8x^4 - 8x^2 + 1}^{4};
    \overbrace{16x^{5}-20x^{3}+5\,x}^{5} \\
    & \underbrace{32x^{6}-48x^{4}+18x^{2}-1}_{6};\underbrace{64x^{7}-112x^{5}+56\,x^{3}-7\,x}_{7};\ \ldots
\end{align*}

They fulfil the second order recursion:
\colorbox{shadecolor}{$T_{n+1}(x)=2xT_n(x)-T_{n-1}(x)\quad(n\geq 2)$}.
We can \emph{normalise the polynomials}: $\frac{1}{2^n}T_{n+1}(x)$,
which is called a min-max polynomial which we can also write in the Newton polynomial form $(x-x_0)\cdots(x-x_n)=\pi_{n+1}(x)$
and has $\max_{-1\leq x \leq 1}|\pi_{n+1}(x)|=\frac{1}{2^n}$.

\paragraph{Discrete Orthogonality}

The Chebyshev polynomials are orthogonal in the following sense:
\begin{align*}
    \langle T_{j},T_{k}\rangle := \sum_{i=0}^{N}T_{j}(x_{i})T_{k}(x_{i})
    =
    \left\{
    \begin{matrix}
        0       & j \neq k \\
        (N+1)/2 & j = k \neq 0 \\
        N+1     & j=k=0
    \end{matrix}
    \right. \\
    {\color{gray} (j,k=0,\ldots,N)}
\end{align*}
\begin{align*}
    x_{i}=\cos\!\left({\frac{2i+1}{2(N+1)}}\pi\right)\quad{\color{gray} (i=0,1,...,N)}
\end{align*}

which results in a diagonal $G^TG$ matrix:
\begin{align*}
    \begin{bmatrix}
        N+1 & 0             & 0             & \ldots & 0             \\
        0   & \frac{N+1}{2} & 0             & \ldots & 0             \\
        0   & 0             & \frac{N+1}{2} & \ldots & 0             \\
        0   & 0             & 0             & \ddots & 0             \\
        0   & 0             & 0             & \ldots & \frac{N+1}{2}
    \end{bmatrix}_{\color{gray}(m+1)\times(m+1)}
\end{align*}

We can use an affine transformation $x\mapsto a + \frac{b-a}{2}(x+1)$ to map the
Chebyshev arguments on an interval $[a,b]$ and it's inverse
$-1+2\frac{x-a}{b-a}$ mapping $[a,b]$ to $[-1, 1]$, resulting in the basis
\begin{align*}
    \{1,T_{1}(-1+2\frac{x-a}{b-a}),T_{2}(-1+2\frac{x-a}{b-a}),T_{3}(-1+2\frac{x-a}{b-a}),\ldots\}
\end{align*}

and the Chebyshev arguments
\begin{align*}
    x_{i}=a+\frac{b-a}{2}\Biggl(\cos\left(\frac{2i+1}{2(N+1)}\pi\right)+1\Biggr)\ \ (i=0,1,...,N)
\end{align*}

\paragraph{Continuous Orthogonality}

With a (theoretical) infinite amount of measurements $N\to\infty$,
the Chebyshev polynomials fulfil the continuous orthogonal relation:
\begin{align*}
    \left\langle T_{j},T_{k}\right\rangle_{cont}
    := \int_{-1}^{1} T_{j}(x)T_{k}(x)\frac{d x}{\sqrt{1-x^{2}}} =
    \left\{
    \begin{matrix}
        0       & j \neq k     \\
        \pi / 2 & j = k \neq 0 \\
        \pi     & j = k = 0
    \end{matrix}
    \right.
\end{align*}

This means, if $y(x)$ is a function on $(-1,1)$ which is absolutely square-integrable with respect to the
weight function $w(x):={\frac{1}{\sqrt{1-x^{2}}}}\quad(x\in(-1,1))$ in the sense that
$\int_{-1}^1|y(x)|^{2}{\frac{d x}{{\sqrt{1-x^{2}}}}<\infty}$
then the continuous square-sum of residuals
\begin{align*}
    S := \int_{-1}^1\biggl(y(x)-\sum_{j=0}^{m}a_{j}T_{j}(x)\biggr\}^{2}\frac{d x}{\sqrt{1-x^{2}}}
\end{align*}
is minimal iif
\begin{align*}
    a_{j} =
    \left\{
    \begin{matrix}
        \frac{2}{\pi}\int_{-1}^{1}y(x)T_{j}(x)\frac{d x}{\sqrt{1-x^{2}}} & j > 0 \\
        \frac{1}{\pi}\int_{-1}^{1}y(x)\frac{d x}{\sqrt{1-x^{2}}}         & j = 0
    \end{matrix}
    \right.
\end{align*}

\subsection{Multivariate Linear Least-Squares Approx.}

X-Arguments now are vectors:
$\vec{x}_0, \vec{x}_1, \ldots, \vec{x}_N$ of dimension $d$.
Usually, tensor products of univariate basis functions are used:
\begin{align*}
    \sum_{j_1=0}^{m_1} \sum_{j_2=0}^{m_2}\cdots \sum_{j_d=0}^{m_d}
    a_{j_1, j_2, \ldots, j_d}g_{j_1}(x^{(1)})g_{j_2}(x^{(2)})\cdots g_{j_d}(x^{(d)})
\end{align*}
defining a basis of $m_1\times m_2\times \cdots \times m_d$ functions in d variables $x^{(1)}, x^{(2)},...,x^{(d)}$.
\\[1em]
\textbf{Example:}
We have a 2d input $(x,y)=\vec{x}_i$ and a 1d output $z_i$.
The residuals we want to minimise therefore are $r_i=z_i-f(\vec{x}_i)$ with $f$ being the approximated surface.
We chose the basis functions ${\color{blue}\{1, x\}}\times{\color{green}\{1, y, y^2\}}$ giving 6 bivariate basis functions:
$\{{\color{blue}1}, {\color{blue}1}{\color{green}y}, {\color{blue}1}{\color{green}y^2},
    {\color{blue}x}\cdot{\color{green}1}, {\color{blue}x}{\color{green}y}, {\color{blue}x}{\color{green}y^2}
\}$ with 6 unknown weight variables $a_{00}, a_{01}, a_{02}, a_{10}, a_{11}, a_{12}$ and the design matrix $G$:

\begin{center}
    \begin{tabular}{l|cccr}
        $\mathbf{G}$ & $g_0 = 1$ & $g_1 = y$        & \ldots & $g_m=xy^2$       \\
        \hline
        $\vec{x}_0$  & 1         & $g_1(\vec{x}_0)$ & \ldots & $g_m(\vec{x}_0)$ \\
        $\vec{x}_1$  & 1         & $g_1(\vec{x}_1)$ & \ldots & $g_m(\vec{x}_1)$ \\
                     &           &                  & \vdots &                  \\
        $\vec{x}_N$  & 1         & $g_1(\vec{x}_N)$ & \ldots & $g_m(\vec{x}_N)$ \\
    \end{tabular}
\end{center}
