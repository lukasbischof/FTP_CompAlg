\section{Least-Squares Approximation}

\begin{itemize}
	\item Error = Residuum, errors = residuals. $e_i=r_i=y_i - \sum_{j=0}^ma_jg_j(x_i)$
	\item $\min_i\sum_i|e_i|$: $L^1$ norm
	\item $\min_i\sum_ie_i^2$: $L^2$ norm
	\item Basis functions $g_0,g_1,\ldots,g_m$ (m=degree, (polynom $x^m$))
	\item normalisation of data: $\frac{x-\mu}{\sigma}$ ($\mu$=mean, $\sigma$=deviation)
\end{itemize}

Some basis functions are:
\begin{itemize}
	\item{
		Trigonometric:
		\begin{align*}
			\left\{e^{i k x}\ |\ k\in Z\right\}{\text{or }}\left\{{\cos(k x){\mathrm{,}}\sin(k x)\ |\ k\in Z}\right\}\quad[x\in(0,2\pi)]
		\end{align*}
	}
	\item{
		Polynomials:
		\begin{align*}
			& \left\{x^{j}\ |\ j\in \mathbb{N}_{0}\right\} & \text{(standard monomials)} \\
			& \left\{\left({\frac{x-\mu}{\sigma}}\right)^{j}|\ j\in \mathbb{N}_{0}\right\} & \text{(normalised standard monomials)} \\
			& \left\{ \pi_{j}(x)\ |\ j\in \mathbb{N}_{0} \right\} & \text{(Newton polynomials)} \\
			& \left\{T_{j}(x)\ |\ j\in \mathbb{N}_{0}\right\} & \text{(Chebyshev, }x\in(-1,1))
		\end{align*}
		\begin{align*}
			& \left\{L_{j}(x)=\frac{e^{x}}{j!}\frac{d^{j}}{d x^{j}}\Big(e^{-x}x^{j}\Big)\ |\ j\in \mathbb{N}_{0}\right\}
			\text{(Laguerre, }x\in(0,\infty)) \\
			& \left\{H_{j}(x)=(-1)^{j}e^{\frac{x^{2}}{2}}\,{\frac{d^{j}}{d x^{j}}}\left(e^{-\frac{x^{2}}{2}}\right)\ |\ j\in \mathbb{N}_{0}\right\} \\
			&\quad\text{(Hermite, }x\in(0,\infty))
		\end{align*}
	}
	\item{
		Exponentials
		\begin{align*}
			\left\{ e^{\alpha kx}\ |\ k\in\mathbb{Z} \right\}
		\end{align*}
	}
\end{itemize}

\subsection{Design Matrix}
Given a \emph{measurement} with $N-1$ points $\{(x_i,y_i)\}_{i=0,\ldots,N}$ (with $x$ a vector in the multivariate case)
and a set of \emph{basis functions} $\{g_j\}_{j=0,\ldots,m}$,
the design matrix $G$ results from the linear system
$y_i=\sum_{j=0}^ma_jg_j(x_i)\quad(i=0,\ldots,N)$ with $N+1$ equations and $m+1$ unknowns:

\begin{snugshade*}
  \begin{align*}
    \underbrace{
      \begin{bmatrix}
        g_0(x_0) & \hdots & g_m(x_0) \\
        \vdots & \ddots & \vdots \\
        g_0(x_N) & \hdots & g_m(x_N)
      \end{bmatrix}
    }_{\text{Design matrix }G}
    \cdot
    \begin{bmatrix}
      a_0 \\
      \vdots \\
      a_m
    \end{bmatrix}
    =
    \begin{bmatrix}
      y_0 \\
      \vdots \\
      y_N
    \end{bmatrix}
  \end{align*}
\end{snugshade*}
that in general is overdetermined ($m<N$) thus the squared sum of residuals
$S=\sum_{i=0}^{N}\left(y_{i}-\sum_{j=0}^{m}a_{j}g_{j}(x_{i})\right)^{2}$
has to be minimised [$\downarrow$].

\paragraph{Normal Equation} Theorem: The squared sum of residuals $S$ is minimal iff:
\colorbox{shadecolor}{$
	G^T\cdot G\cdot a| = G^T\cdot y|\quad (a|\text{ and }y|\text{ are the column vectors})
$}

with $G^T\cdot G$ being a normal,
symmetric and relatively small $(m+1)\times(m+1)$ matrix resulting in $f(x)=\sum a_jg_j(x)$ as best $L^2$ fit for data.

\subsection{Singular Value Decomposition}
Any real matrix $G$ of dimensions $((N+1)\times (m+1))$ can be decomposed as the product $G=UDV^T$ 
where $U$ is an orthogonal (meaning $UU^t=U^tU=I\Rightarrow U^{-1}=U^T$) (large) matrix with $\dim U = (N+1)\times (N+1)$, 
D is a diagonal matrix with $\dim D = (N+1)\times (m+1)$ and
V is orthogonal (small) with $\dim V = (m+1)\times (m+1)$.

Using this decomposition, the minimisation of the square sum of residuals can be found in a numerically stable way:
With $G$ decomposed to $G=UDV^T$, the system translates to
\colorbox{shadecolor}{
	$\hat{r}| = \hat{y}| - D\hat{a}| = \vec{0}\quad$
	{\color{darkgray} $\left[\hat{r} = U^Tr|,\ \hat{y}| = U^Ty|\text{ and }\hat{a} = V^Ta|\right]$}
}

and is solved for zero since $\hat{S}$ is minimal iff $\hat{r}_i=0\quad(i=0,\ldots,m)$.
Due to the orthogonality of $U$, the square sum $\hat{S}$ of residuals $\hat{r}$ is equal to the square sum $S$ of the original
residuals $r|$.

In the final step, we receive $a| = V\cdot\hat{a}|$.

\subsection{Uniform Arguments and Orth. Polynomials}

If the set of basisi functions $\{g_j\}_{j=0,\ldots,m}$ is \emph{orthogonal} with respect to the inner product
$\langle g_j,g_k\rangle := \sum_{i=0}^N g_j(x_i)g_k(x_i)$, $G^TG$ is diagonal and the solution is trivial:
\begin{align*}
	a_j=\frac{\langle y|, g_j|\rangle}{\langle g_j|, g_j|\rangle}\quad(j=0,1,\ldots,m)
\end{align*}

\paragraph{Uniform Arguments}
If we have a set of uniformly distributed arguments of a measurement $\{x_0 + t\cdot h\}_{t=0,\ldots,N}$,
then is the set of polynomials
\begin{align*}
	p_{k,N}(t) 
	= \sum_{i=0}^{k}(-1)^{i}{\binom{k}{i}}{\binom{k+i}{i}}{\frac{t^{(i)}}{N^{(i)}}}\quad(k=0,1,...,N)
\end{align*}
with integer normalisation $t = (x-x_0)\div h$ orthogonal with respect to the inner product $\langle g,\tilde{g}\rangle$.
Here, $N^{(i)} = N(N-1)(N-2)\cdots(N-i+1)$ and $t^{(i)} = t(t-1)(t-2)\cdots(t-i+1)$ are defined as falling factorials.

$\Rightarrow$ We therefore get a design matrix with ``x''-values $p_{0,N}(t), p_{1,N}(t),\ldots$
(e.g. $1$; $1-\frac{t}{2}$; $1-\frac{3t}{2}+\frac{1}{2}(-1+t)t$, ...) and ``y''-values being the $t=0,\ldots,N$ integer values.

























